\documentclass{article}

\usepackage[utf8]{inputenc}
\usepackage{graphicx}
\usepackage{amsmath}
\usepackage{amssymb}

\usepackage{mathrsfs}
\pagestyle{empty}
\DeclareMathAlphabet{\mathpzc}{OT1}{pzc}{m}{it}


\usepackage{setspace}

% Meta info
\author{Martin Hansen}
\title{Matematik 1}

% Math section related
\newcommand{\cent}[1]{\begin{center}#1\end{center}}
\newcommand{\mAlign}[1]{\begin{align*}#1\end{align*}}
\newcommand{\mat}[1]{\begin{gather*}#1\end{gather*}}

% Macro section
\newcommand{\afb}[3]{\ensuremath{_#1 \textbf{#2}_#3}}
\newcommand{\vek}[3]{\ensuremath{\begin{bmatrix} #1\\ #2\\ #3\end{bmatrix}}}
\newcommand{\vekt}[2]{\ensuremath{\begin{bmatrix} #1\\ #2\end{bmatrix}}}
\newcommand{\mediumMatrix}[9]{\ensuremath{
		\begin{bmatrix}
			#1 & #2 & #3 \\
			#4 & #5 & #6 \\
			#7 & #8 & #9
		\end{bmatrix}}}
\newcommand{\smallMatrix}[4]{\ensuremath{\begin{bmatrix}
			#1 & #2 \\
			#3 & #4
\end{bmatrix}}}

\newcommand{\script}[1]{\mathpzc{#1}}

% Remove "New paragraph" indentation
\setlength{\parindent}{0cm} 

\begin{document}
	
	\maketitle
	\begin{center}
		\Large \textbf{Forord}
		
		Denne formelsamling baserer sig på DTU eNotes E1005.
	\end{center}
	\pagebreak
	\tableofcontents
	\pagebreak
	\part{\textbf{Geometri i det euklide rum}}
	
	\section{Egenværdiproblemet}
	
	\subsection*{Introduktion}
	
	For en vilkårlig afbildning $f$, kan det være tilfældet, at en vektor er proportional med sin billedvektor. En sådan billedvektor kan så være skaleret i forhold til den vektor den afbilder, og den vil så kunne beskrives af et produkt mellem en vektor og en skalar $\lambda$. Et sådan produkt beskriver et kendt problem der går under navnet \textit{egenværdisproblemet}. Det er dette emne der behandles i denne sektion, og det er en vigtig gren indenfor ingeniørmatematikken. \newline
	
	I dette kapitel behandles afbildninger der afbilder et objekt i $V$ ind i $V$. Dvs. en afbildning defineret ved nedenstående:
	
	\cent{$ f:V \rightarrow V $}
	
	hvor det ses at definitionsrummet og dispositionsrummet er ens og hvor en vektor som er identisk med sin afbildning kaldes en fiksvektor. 
	
	\subsection{Egenværdier og egenvektorer}
	
	Egenværdiproblemet er et synonym for det problem der omhandler vektorer og proportionalitet med hensyn til deres billedvektor. Der gælder det helt særlige, at vektorer kan multipliceres med skalarer og returnere en skalaret vektor. Nedenstående afbildning kan betragtes:
	
	\cent{$ f(v)=\lambda v $}
	
	Hvor skalaren $\lambda$ er en egenværdi for en given afbildning $f(v)$ og $v$ er afbildningens dertilhørende egenvektor. 
	
	\subsubsection{Eksplicit aflæsning af egenværdier ud fra afbildningsmatrix}
	
	
	Lad $Gr_3(\mathbb{R}$ betegne et vektorrum af vektorer, så kan der gives en lineærafbildning $f$:
	
	\cent{$f : Gr_3(\mathbb{R})\mapsto Gr_3(\mathbb{R})$ }
	
	der med hensyn til en given basis $\textbf{a}=(a_1,a_2,a_3)$ har en dertilhørende afbildningsmatrix \afb{a}{F}{a}:
	
	\cent{$ \afb{a}{F}{a} = \mediumMatrix{2}{0}{0}{0}{4}{0}{0}{0}{6} $}
	Afbildes $a_1$, $a_2$ og $a_3$ i $f$ fås:
	
	\cent{$ f(a_1)=\ \afb{a}{F}{a} \ _a a_1 = \mediumMatrix{2}{0}{0}{0}{4}{0}{0}{0}{6} \cdot \vek{1}{0}{0} = \vek{2}{0}{0} $}
	\cent{$ f(a_2)=\ \afb{a}{F}{a} \ _a a_2 = \mediumMatrix{2}{0}{0}{0}{4}{0}{0}{0}{6} \cdot \vek{0}{1}{0} = \vek{0}{4}{0} $}	
	\cent{$ f(a_3)=\ \afb{a}{F}{a} \ _a a_3 = \mediumMatrix{2}{0}{0}{0}{4}{0}{0}{0}{6} \cdot \vek{0}{0}{1} = \vek{0}{0}{6} $}
	
	Her ses det at de indeholdte vektorer i basisen $a$ er egenvektorer til afbildningsvektorerne i $f$, hvor koefficienten til hhv. $a_1$, $a_2$ og $a_3$ er afbildningernes egenværdi tilhørende deres egenvektor. Egenværdierne er her givet som diagonalelementerne i \afb{a}{F}{a}, men dette er ikke altid tilfældet.
	
	\subsubsection{Geometrisk fortolkning}
	I ovenstående tilfælde kan det ses, at summen af de tre basisvektorer $ a_1 $, $ a_2 $ og $ a_3 $ er blevet strukket med $\lambda=2$ langs x-aksen. Derefter med $\lambda=4$ langs y-aksen og til sidst med $\lambda=6$ op ad z-aksen. Det er derfor tydeligt, at $a_1$, $a_2$ og $a_3$ må være egenvektorer for $f$ og 2, 4 og 6 er så afbildningsens egenværdier.
	Hvis de tre billedvektorer er lineært uafhængige, kan man også sige, at det parallelepidium som de tre billedvektorer udspænder, er blevet strukket med de ovennævnte egenværdier i hhv. hver af de tre retninger. Det kan indses ved matrixproduktet mellem afbildningsmatricen og en tilfældig vektor med koordinaterne $x_1$, $x_2$ og $x_3$: 
	
	\cent{$ \mediumMatrix{2}{0}{0}{0}{4}{0}{0}{0}{6} \cdot \vek{x_1}{x_2}{x_3} = \vek{2x_1}{4x_2}{6x_3} $}
	
	\subsubsection{Eksempel - egenværdiproblemet for differentiation}
	
	En lineær afbildning $ f : \mathbb{C}^\infty \mapsto \mathbb{C}^\infty $ kan gives ved:
	
	\cent{$ f(x(t)) = x'(t) $}
	
	$x(t)$ kan defineres ved:
	
	\cent{$ x(t)= e^{\lambda t}$}
	
	Dens afledte kan så bestemmes til:
	
	\cent{$x'(t)=\lambda  e^{\lambda t}$}
	
	Hvilket er afbildningen af $ x(t) $ med hensyn til $ f $:
	
	\cent{$ f(e^{\lambda t}) = \lambda  e^{\lambda t} $}
	
	Her ses det, at egenværdien er $\lambda$ og dens egenvektor er $e^{\lambda t}$. Egenværdien kan antage et ubestemt antal værdier og angives $k$; afbildningens egenrum kan så bestemmes ved:
	
	\cent{$E_\lambda = \{ke^{\lambda t} | k \in \mathbb{R}\} $}
	
	\subsubsection{Bestemmelse af egenværdier og egenvektorer ud fra vilkårlig kvadratisk matrice}
	
	I det foregående eksempel kunne egenværdierne for en afbildning eksplicit aflæses i afbildningsmatricen \afb{a}{F}{a}, men det er ikke altid tilfældet. Det er nemlig ikke kun diagonalmatricer der har egenværdier, vilkårlige kvadratiske matricer kan også have egenværdier som ikke kan aflæses direkte ud fra dens elementer. Til disse må man ty til nogle andre matricer, og det kræver noget mere viden omkring matricer og egenværdier. \newline
	Man kan betragte en matrice som er afbildningsmatrix for en afbildning $h$:
	
	\cent{$ _eh = \smallMatrix{\frac{7}{3}}{\frac{2}{3}}{\frac{1}{3}}{\frac{8}{3}}$}
	
	To vektorer kan gives ved $v_1=(2,-1)$ og $v_2=(1,1)$, begge med basis i $e$. Vektorproduktet mellem $_eh$ og hhv. $v_1$ og $v_2$ kan bestemmes til:
	
	\cent{$ _eh(v_1)= \smallMatrix{\frac{7}{3}}{\frac{2}{3}}{\frac{1}{3}}{\frac{8}{3}} \cdot \vekt{2}{-1} = \vekt{4}{-2} = 2 \cdot \vekt{2}{-1}$}
	\cent{$ _eh(v_2)= \smallMatrix{\frac{7}{3}}{\frac{2}{3}}{\frac{1}{3}}{\frac{8}{3}} \cdot \vekt{1}{-1} = \vekt{3}{3} = 3 \cdot \vekt{1}{1}$}
	
	Her ses det, at 2 og 3 kan være afbildningens egenværdier, og de to vektorer $v_1=(2,-1)$ og $v_1=(1,1)$ egenværdiernes dertilhørende egenvektorer.
	
	\subsubsection{Egenværdier og egenvektorer for matricer}
	Alt indhold i de foregåedne sektioner gælder også for matricer og definitionen for egenværdiproblemet med hensyn til matricer er også identisk med definitionen for vektorer:
	
	\cent{$ \textbf{Av} = \lambda \textbf{v}, \quad \textbf{v} \neq \textbf{o} $}
	
	Hvor $ \textbf{A} $ er en vilkårlig kvadratisk matrice, $ \lambda $ er en egenværdi for $\textbf{A}$ med dens dertilhørende egenvektor $ \textbf{v} $.
	
	\subsubsection{Undersøge om vektorer er egenvektorer for en matrix}
	
	Der er givet et sæt af vektorer $ v $ som indeholder tre vektorer $ ((2,3),(4,4),(2,-1)) $. Der skal undersøges om disse er egenvektorer for nedenstående matrix:
	
	\cent{$ \textbf{A} = \smallMatrix{4}{-2}{3}{-1} $}
	
	Det kan gøres ved hjælp af matrix-vektor produktet mellem hver af en vektorer i sættet og $ \textbf{A} $:
	
	\mAlign{\textbf{A} \textbf{v}_1 &= \smallMatrix{4}{-2}{3}{-1} \cdot \vekt{2}{3} = \vekt{2}{3} = \textbf{v}_1 \\
		\textbf{A} \textbf{v}_1 &= \smallMatrix{4}{-2}{3}{-1} \cdot \vekt{4}{4} = \vekt{8}{8} = 2\textbf{v}_2 \\ 
		\textbf{A} \textbf{v}_1 &= \smallMatrix{4}{-2}{3}{-1} \cdot \vekt{2}{-1} = \vekt{10}{7} \neq \lambda \textbf{v}_3}
	
	De to første vektorer er egenvektorer for $ \textbf{A} $ da de resulterende vektorer er proportionale med $ \textbf{A} $, den sidste derimod, er ikke proportional og er derved ikke en egenvektor for $ \textbf{A} $.
	
	\subsection{Generelle egenskaber for egenvektorer og egenværdier}
	\subsubsection{Egenvektorrummet}
	I en senere sektion vil det også vise sig, at mængden af egenvektorer for en egenværdi er uendelig. Denne mængde vil så også udgøre et underrum i $V$ da det er stabilt med hensyn til addition og multiplikation. Dette underrum kaldes også \textit{egenrummet}, og hvis dette rum er endeligt dimensionalt, så omtales det som den \textit{geometriske multiplicitet af $\lambda$}, $gm(E_\lambda)$. Dette kan matematisk også udtrykkes som:
	
	\cent{$ E_\lambda = \{ \textbf{v} \in V | f(\textbf{v} = \lambda \textbf{v}) \}$}
	\cent{$ gm(E_\lambda) = dim(E_\lambda), \quad dim(E_\lambda)\neq \infty $}
	
	Dette gælder for både vektorer og matricer med både reelle såvel komplekse egenværdier og egenvektorer.
	
	Kort gengivet:
	
	\begin{itemize}
		\item Mængden af egenvektorer tilhørende en vilkårlig egenværdi er uendelig.
		\item Denne mængde udgør også et underrum af $V$ der også omtales som \textit{egenrummet}.
		\item Hvis denne er endelig dimensional kaldes det også \textit{den geometriske multiplicitet af $\lambda$}.
		\item Gælder både for reelle og komplekse egenværdier og egenvektorer.
		\subitem Disse egenværdiers tilhørende egenrum vil så udgøre et underrum i hhv. $ \mathbb{R}^n $ og $ \mathbb{C}^n $
	\end{itemize}
	\subsubsection{Egenrummet og dens egenskaber}
	For en afbildning $f : V \mapsto V$ gælder der det særlige omkring udvælgelse af egenvektorer i de forskellige egenrum der knytter sig til afbildningens egenværdier at:
	\begin{itemize}
		\item $f$ har en række egenværdier med dertilhørende egenrum.
		\item For hvert af disse egenrum, kan der udvælges nogle lineært uafhængige vektorer.
		\item Disse kan sammenfattes til et vektorsæt.
	\end{itemize}
	Disse vektorer vil så være lineært uafhængige.
	\subsubsection*{Bevis}
	
	Ud fra antagelsen om at ovenstående ikke er tilfældet, må det gælde at en vektor $ \textbf{x} $ kan hives ud af sættet og beskrives af det udtyndede sæt, hvor alle lineært afhængige vektorer er pillet ud: 
	
	\cent{$ \textbf{x} = k_1v_1+k_2v_2+...+k_mv_m $}
	
	Alle de led hvor der indgår en nul-koefficient antages at være pillet ud. \newline
	
	Denne vektor, samt alle de andre vektorer i linearkombinationen, har nogle dertilhørende egenværdier tilknyttet, og disse  egenværdier for hhv. $\textbf{x}$ og $v_i$ betegnes $\lambda$ og $\lambda_i$. Et udtryk for $\lambda\textbf{x}$ kan opnås dels ved at gange igennem med $\lambda$ eller bestemme billedet af $f$ på begge sider:
	
	\cent{$\lambda\textbf{x} = \lambda k_1 v_1+\lambda k_2 v_2+...+\lambda k_m v_m $}
	\cent{$\lambda\textbf{x} = \lambda_1 k_1 v_1+\lambda_2 k_2 v_2+...+\lambda_m k_m v_m $}
	
	Da udtrykkende er de samme kan nul-vektoren beskrives ved at trække de to ligninger fra hinanden:
	
	\cent{ $ \textbf{0} = k_1(\lambda - \lambda_1)v_1 + k_2(\lambda - \lambda_2)v_2 + ... + k_m(\lambda - \lambda_m)v_m $ }
	
	Da alle nul-koefficienter var pillet ud, må det så betyde at egenværdien for $ \textbf{x} $ må være lig med egenværdierne for $v_i$, men dette må så betyde, at vektorerne må være valgt ud fra samme egenrum, og derfor må være lineært uafhængige. Dette strider også mod at $\textbf{x}$ er en linearkombination af basisvektorer, da det ikke er muligt hvis den er del af et lineært uafhængigt sæt. \newline
	
	Hvis ovenstående er korrekt, må mindst en af koefficienterne ($(\lambda-\lambda_i)$) være forskellig fra 0, men så vil nul-vektoren $\textbf{0}$ være beskrevet ved en egentlig linearkombination af vektorer, hvilket strider mod reglerne vedrørende lineær uafhængighed.\newline
	
	Derfor må antagelsen nødvendigvis føre til modstrid.
	
	\subsubsection{Egenbasis}
	
	Det førnævnte vektorsæt i forrige sektion har den egenskab, at hvis summen af de geometriske multipliciter for hver egenværdi er lig med dimensionen af $V$, så er vektorsættet en basis for dette vektorrum. En sådan basis kaldes også for en \textit{egenvektorbasis}; eller kort: \textit{egenbasis}.
	
	\subsubsection{Opsummering}
	På baggrund af de foregående sektioner og mere generelt, kan følgende liste af egenskaber opstilles.\newline
	
	For en lineær afbildning $f : V \mapsto V$ ind i sig selv med dimension $n$, gælder der:
	\begin{itemize}
		\item Egentlige egenvektorer som hører til forskellige egenværdier for $f$, er lineært uafhængige.
		\item $f$ kan højst have $n$ forskellige egenværdier.
		\item Hvis $f$ har $n$ forskellige egenværdier, findes der en basis for $V$ bestående af egenvektorer for $f$.
		\item Summen af de geometriske multipliciteter af egenværdierne for $f$ kan højst være $n$.
		\item 	Hvis, og kun hvis, summen af de geometriske multipliciteter af egenværdierne for $f$ er lig med dimensionen af $V$, findes der en basis for $V$ bestående af egenvektorer for $f$. En sådan basis kaldes også for en \textit{egenbasis}.
	\end{itemize}
	\subsection{Afbildningsmatricer i forhold til egenbasis}
	
	Lad $ f : V \mapsto V $ betegne en lineær afbildning af et $n$-dimensionalt vektorrum V ind i sig selv, og lad $ v=(v_1,…,v_n)$ være en basis for $V$. Der gælder da:
	
	\begin{enumerate}
		\item Afbildningsmatricen $\afb{v}{F}{v}$ for $f$ med hensyn til $v$ er en diagonalmatrix, hvis, og kun hvis, $v$ er en egenbasis for $V$ med hensyn til $f$.
		\item Antag, at $v$ er en egenbasis for $V$ med hensyn til $f$. Lad $\Lambda$ betegne den diagonalmatrix, som er afbildningsmatrix for $f$ med hensyn til $v$. Rækkefølgen af diagonalelementerne i $\Lambda$ er da bestemt ud fra den valgte basis således: Basisvektoren $v_i$ hører til den egenværdi $\lambda_i$, som står i den i’te søjle i $\Lambda$.
	\end{enumerate}
	
	\subsubsection{Eksempel på diagonal afbildningsmatrix med reelle elementer}
	
	En lineær afbildning $ f : \mathbb{R}^2 \mapsto \mathbb{R}^2 $ er givet ved:
	
	\mAlign{f(a)&=a_1 \\ f(a)&=-a_2}
	
	Egenværdierne kan tydeligt bestemmes til 1 og -1 og afbildningsmatricen for $ f $ kan uden videre angives ved:
	
	\cent{ $ \afb{e}{f}{e} = \smallMatrix{1}{0}{0}{-1}$}
	
	Denne afbildningsmatrix, som er en diagonalmatrix, har afbildningens egenværdier som diagonalelementer.
	
	\subsubsection*{Eksempel på diagonal afbildningsmatrix med komplekse elementer}
	Lad $ f : \mathbb{C}^2 \mapsto \mathbb{C}^2 $ være en kompleks lineær afbildning som opfylder:
	\cent{$ f(\script{z}_1,\script{z}_2) = (-\script{z}_2,\script{z}_1) $}
	
	Afbildningen af de komplekse vektorer $(i,1)$ og $(-i,1)$ kan bestemmes til følgende:
	
	\cent{$ f(i,1)=(-1,i)=i(i,1) $}
	\cent{$ f(-i,1)=(-1,-i)=-i(-i,1) $}
	
	Det ses, at $i$ og $-i$ er egenværdier for $f$ med tilhørende egenvektorer $(i,1)$ og $(-i,1)$.  Da vektorerne ovenikøbet er lineært uafhængige, er det også en egenbasis for vektorrummet $\mathbb{C}^2$ med tilhørende afbildningsmatrix som givet ved nedenstående:
	
	\cent{$ \Lambda = \smallMatrix{\lambda_1}{0}{0}{\lambda_2} = \smallMatrix{i}{0}{0}{-i} $}
	
	
	
	\subsection{Specielt om matricer}
	\subsubsection{En generel bestemmelse af deres egenværdier og egenvektorer}
	
	De metoder der indtil videre er benyttet, kan beskrives som værende manuelle og trivielle og det har ofte indebåret valg af vektorer der i sidste ende ikke var tilfældigt udvalgt. Til bestemmelse af egenværdier og deres egenvektorer ved hjælp af en mere generel metode, bliver man nød til at omskrive på definitionen for egenværdiproblemet. Her benyttes at $ \textbf{Ev}=\textbf{v} $:
	
	\mat{\textbf{Av} = \lambda \textbf{v} 
		 \\ \Leftrightarrow  \\
		  \textbf{Av} - \lambda \textbf{v} = 0 
		  \\ \Leftrightarrow \\
	  		\textbf{Av} - \lambda (\textbf{Ev}) = 0 \\
	  		\Leftrightarrow \\
  				\textbf{Av} - (\lambda \textbf{E})\textbf{v} = 0 \\ 
  				\Leftrightarrow \\
  					(\textbf{A}-\lambda \textbf{E})\textbf{v}=0}
  	
  	Den sidste ligning som er fremkommet ved omskrivningen indeholder en matrix der går under tilnavnet \textit{Den Karakteristiske Matrix}. Denne ligning er et homogent ligningssystem med \textit{n} ligninger (antallet af rækker i \textbf{A}) og \textit{n} ubekendte (dimensionen af \textbf{v}). Dette ligningssystem kan have to forskellige typer af løsninger, enten er nul-vektoren en løsning ellers findes der et uendeligt antal af løsninger. Da en egenvektor skal være en \textit{egentlig} vektor, så skal der kunne bestemmes en egenvektor ud fra en uendelig mængde af løsninger. Skrevet på en anden måde, \textit{den karakteristiske matrix} skal være singulær, og på baggrund af dette kan determinanten af den karakteristiske matrix bestemmes ved:
  	
  	\cent{$ det(\textbf{A}-\lambda \textbf{E}) $}
  	
  	Determinanten af denne matrix vil så returnere et reelt polynomium af en grad der modsvarer dimensionen af \textit{Den Karakteristiske Matrix}, og hvis løsninger er egenværdier for matricen \textbf{A}. Her skal man dog lige stoppe op og overveje hvad det indebærer, at man har med et reelt polynomium at gøre. For reelle polynomier gælder der for f.eks. komplekse løsninger, at de altid optræde i såkaldte \textit{konjugerede par}, dvs. hvis $  \script{z} =1 + i $ er en løsning, så er $ \bar{\script{z}} =1 - i $ også en løsning. Derved gælder, at komplekse egenværdier også vil optræde i konjugerede par.\\
  	\\
  	Dette polynomium kaldes også \textit{Det Karakteristiske Polynomium} og sættes denne lig 0 fås \textit{karakterligningen}:
  	
  	\cent{$ det(\textbf{A}-\lambda \textbf{E}) = 0 $}
  	
  	Fremadrettet kan man referere til ovenstående som \textit{De tre Karakteristiske trin} for nemhedens skyld.
  	
  	\subsubsection{Fremgangsmåde ved brug af De Tre Karakteristiske Egenskaber}
  	
  	Bestemmelse af egenværdier og egenvektorer for vilkårlige kvadratiske matricer ved brug af \textit{De tre Karakteristiske trin} foretages ved følgende tre trin:
  	\begin{enumerate}
  		\item  Bestem \textit{Den Karakteristiske Matrix}.
  		\item Bestem determinanten af denne og derefter løsningerne for \textit{Det Karakteristiske Polynomum}.
  		\item Indsæt løsningerne i \textit{Den Karakteristiske Matrix} og GaussJordan reducer denne.
  	\end{enumerate}
  	
	\subsubsection{Bestemmelse af egenværdier og egenvektorer ud fra eksempel}
	
	Givet en matrice \textbf{A}:
	
	\cent{$ \textbf{A} = \smallMatrix{\frac{7}{3}}{\frac{2}{3}}{\frac{1}{3}}{\frac{8}{3}} $}
	
\end{document}